\documentclass[10pt,a4paper]{article}
\usepackage[utf8]{inputenc}
\usepackage[italian]{babel}
\usepackage{amsmath}
\usepackage{gensymb}
\usepackage{subfig}
\usepackage{amsfonts}
\usepackage{amssymb}
\usepackage{wrapfig}
\usepackage{graphicx}
\usepackage{xcolor}
\usepackage{float}
\usepackage{booktabs}
\usepackage[left=2cm,right=2cm,top=2cm,bottom=2cm]{geometry}
\newcommand{\rem}[1]{[\emph{#1}]}
\newcommand{\exn}{\phantom{xxx}}
\renewcommand{\thesubsection}{\thesection.\alph{subsection}}  %% use 1.a numbering
\author{Gruppo E.B24 \\ Giovanni Sucameli, Francesco Sacco, Davide Incalza}
\title{Esperienza di Fisica: Ottica I}
\begin{document}
	\date{9 Maggio 2019}
	\maketitle
    \begin{center}
		\subsection*{Introduzione}
		L'esperienza si articola in due fasi: nella prima verrà impiegato uno spettroscopio a prisma (opportunamente tarato) per la misura della lunghezza d'onda di una riga spettrale del sodio. Nella seconda verrà valutata la risoluzione di uno spettroscopio a reticolo, impiegato successivamente per la misura della costante di Rydberg sfruttando le righe di emissione dell'idrogeno.
	\end{center}


\section*{PARTE II: Misura della Costante di Rydberg}
	\subsection*{Lampada al Sodio}
		


\end{document}